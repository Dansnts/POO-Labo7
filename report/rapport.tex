\documentclass[12pt, a4paper, onecolumn]{article} 
\usepackage[utf8]{inputenc}
\usepackage{fancyhdr}
\usepackage{graphicx}
\usepackage{geometry}
\usepackage{float}
\usepackage{multicol} 
\usepackage{lmodern}  
\usepackage[T1]{fontenc}  
\usepackage{hyperref}  
\usepackage[dvipsnames]{xcolor}  % Enable predefined colors
\usepackage{listings, lstautogobble}
\usepackage{listings}


\definecolor{codegreen}{rgb}{0,0.6,0}
\definecolor{codegray}{rgb}{0.5,0.5,0.5}
\definecolor{codepurple}{rgb}{0.58,0,0.82}
\definecolor{backcolour}{rgb}{0.95,0.95,0.92}

\lstdefinestyle{mystyle}{
	backgroundcolor=\color{backcolour},   
	commentstyle=\color{codegreen},
	keywordstyle=\color{magenta},
	numberstyle=\tiny\color{codegray},
	stringstyle=\color{codepurple},
	basicstyle=\ttfamily\footnotesize,
	breakatwhitespace=false,         
	breaklines=true,                 
	captionpos=b,                    
	keepspaces=true,                 
	numbers=left,                    
	numbersep=5pt,                  
	showspaces=false,                
	showstringspaces=false,
	showtabs=false,                  
	tabsize=2
}

% Réduire les marges
\geometry{top=1.5cm, bottom=1.5cm, left=1.5cm, right=1.5cm}

%%%%%%%%%%%%%%%%%%%%%%%%%%%%%%%%%%%%%%%%%
% Wenneker Article
% Structure Specification File
% Version 1.0 (28/2/17)
%
% This file originates from:
% http://www.LaTeXTemplates.com
%
% Authors:
% Frits Wenneker
% Vel (vel@LaTeXTemplates.com)
%
% License:
% CC BY-NC-SA 3.0 (http://creativecommons.org/licenses/by-nc-sa/3.0/)
%
%%%%%%%%%%%%%%%%%%%%%%%%%%%%%%%%%%%%%%%%%

%----------------------------------------------------------------------------------------
%	PACKAGES AND OTHER DOCUMENT CONFIGURATIONS
%----------------------------------------------------------------------------------------

\usepackage[english]{babel} % English language hyphenation

\usepackage{microtype} % Better typography

\usepackage{amsmath,amsfonts,amsthm} % Math packages for equations

\usepackage[svgnames]{xcolor} % Enabling colors by their 'svgnames'

\usepackage[hang, small, labelfont=bf, up, textfont=it]{caption} % Custom captions under/above tables and figures

\usepackage{booktabs} % Horizontal rules in tables

\usepackage{lastpage} % Used to determine the number of pages in the document (for "Page X of Total")

\usepackage{graphicx} % Required for adding images

\usepackage{enumitem} % Required for customising lists
\setlist{noitemsep} % Remove spacing between bullet/numbered list elements

\usepackage{sectsty} % Enables custom section titles
\allsectionsfont{\usefont{OT1}{phv}{b}{n}} % Change the font of all section commands (Helvetica)

%----------------------------------------------------------------------------------------
%	MARGINS AND SPACING
%----------------------------------------------------------------------------------------

\usepackage{geometry} % Required for adjusting page dimensions

\geometry{
	top=1cm, % Top margin
	bottom=1.5cm, % Bottom margin
	left=2cm, % Left margin
	right=2cm, % Right margin
	includehead, % Include space for a header
	includefoot, % Include space for a footer
	%showframe, % Uncomment to show how the type block is set on the page
}

\setlength{\columnsep}{7mm} % Column separation width

%----------------------------------------------------------------------------------------
%	FONTS
%----------------------------------------------------------------------------------------

\usepackage[T1]{fontenc} % Output font encoding for international characters
\usepackage[utf8]{inputenc} % Required for inputting international characters

\usepackage{XCharter} % Use the XCharter font

%----------------------------------------------------------------------------------------
%	HEADERS AND FOOTERS
%----------------------------------------------------------------------------------------

\usepackage{fancyhdr} % Needed to define custom headers/footers
\pagestyle{fancy} % Enables the custom headers/footers

\renewcommand{\headrulewidth}{0.0pt} % No header rule
\renewcommand{\footrulewidth}{0.4pt} % Thin footer rule

\renewcommand{\sectionmark}[1]{\markboth{#1}{}} % Removes the section number from the header when \leftmark is used

%\nouppercase\leftmark % Add this to one of the lines below if you want a section title in the header/footer

% Headers
\lhead{} % Left header
\chead{\textit{\thetitle}} % Center header - currently printing the article title
\rhead{} % Right header

% Footers
\lfoot{} % Left footer
\cfoot{} % Center footer
\rfoot{\footnotesize Page \thepage\ of \pageref{LastPage}} % Right footer, "Page 1 of 2"

\fancypagestyle{firstpage}{ % Page style for the first page with the title
	\fancyhf{}
	\renewcommand{\footrulewidth}{0pt} % Suppress footer rule
}

%----------------------------------------------------------------------------------------
%	TITLE SECTION
%----------------------------------------------------------------------------------------

\newcommand{\authorstyle}[1]{{\large\usefont{OT1}{phv}{b}{n}\color{Red}#1}} % Authors style (Helvetica)

\newcommand{\institution}[1]{{\footnotesize\usefont{OT1}{phv}{m}{sl}\color{Black}#1}} % Institutions style (Helvetica)

\usepackage{titling} % Allows custom title configuration

\newcommand{\HorRule}{\color{White}\rule{\textwidth}{5pt}} 
\newcommand{\mydate}{\hfill \textbf{\today}} % Met en gras et aligne à gauche


\pretitle{
	\vspace{-30pt} % Move the entire title section up
	\HorRule\vspace{10pt} % Horizontal rule before the title
	\fontsize{32}{36}\usefont{OT1}{phv}{b}{n}\selectfont % Helvetica
	\color{Red} % Text colour for the title and author(s)
}

\posttitle{\par\vskip 15pt} % Whitespace under the title

\preauthor{} % Anything that will appear before \author is printed

\postauthor{ % Anything that will appear after \author is printed
	\vspace{10pt} % Space before the rule
	\par\HorRule % Horizontal rule after the title
	\vspace{20pt} % Space after the title section
}

%----------------------------------------------------------------------------------------
%	ABSTRACT
%----------------------------------------------------------------------------------------

\usepackage{lettrine} % Package to accentuate the first letter of the text (lettrine)
\usepackage{fix-cm}	% Fixes the height of the lettrine

\newcommand{\initial}[1]{ % Defines the command and style for the lettrine
	\lettrine[lines=3,findent=4pt,nindent=0pt]{% Lettrine takes up 3 lines, the text to the right of it is indented 4pt and further indenting of lines 2+ is stopped
		\color{DarkGoldenrod}% Lettrine colour
		{#1}% The letter
	}{}%
}

\usepackage{xstring} % Required for string manipulation

\newcommand{\lettrineabstract}[1]{
	\StrLeft{#1}{1}[\firstletter] % Capture the first letter of the abstract for the lettrine
	\initial{\firstletter}\textbf{\StrGobbleLeft{#1}{1}} % Print the abstract with the first letter as a lettrine and the rest in bold
}

%----------------------------------------------------------------------------------------
%	BIBLIOGRAPHY
%----------------------------------------------------------------------------------------

\usepackage[backend=bibtex,style=authoryear,natbib=true]{biblatex} % Use the bibtex backend with the authoryear citation style (which resembles APA)

\addbibresource{example.bib} % The filename of the bibliography

\usepackage[autostyle=true]{csquotes} % Required to generate language-dependent quotes in the bibliography
 

%----------------------------------------------------------------------------------------
%	ARTICLE INFORMATION
%----------------------------------------------------------------------------------------

\title{Labo7 - POO \\ Calculatrice} 

% Modifiez le style des auteurs
\newcommand{\largename}[1]{{\Large\textbf{#1}}} % style pour le nom de famille

\author{
	\authorstyle{Dani Tiago \largename{Faria dos Santos}\\ Antoine \largename{Aubry } \\ \\ Groupe  \textbf{L02GrP}\\ HEIG-VD} % Authors
}

\makeatletter
\renewcommand\date[1]{\gdef\@date{\hbox to \linewidth{#1\hss}}}
\makeatother
\date{\today}


\pagestyle{fancy}
\fancyhf{}
\fancyhead[L]{Laboratoire 7 - POO}
\fancyhead[R]{Groupe L03GrD}
\fancyfoot[R]{Page \thepage}
\fancyfoot[L]{HEIG-VD | Dani Tiago \textbf{Faria dos Santos} - Antoine \textbf{Aubry }}



% Ligne sous l'en-tête
\renewcommand{\headrulewidth}{0.7pt}
% Pas de ligne sur le pied de page
\renewcommand{\footrulewidth}{0.5pt}


\begin{document}
	% Utilisation de twocolumn pour le titre
	\twocolumn[ 
	\maketitle
	]
	
	% Retour à une seule colonne
	\onecolumn 
	
	% Table des matières
	\tableofcontents
	\newpage
	
	
	\section{Choix de conception}
\subsection{Operator}

\begin{flushleft}
	Pour la classe \texttt{Operator}, nous avons décidé de créer une hiérarchie en concevant des sous-classes abstraites pour chaque type d'opération. Cela permet de structurer et d'optimiser les opérations selon le principe de la séparation des responsabilités. Par exemple, la méthode \texttt{Digit()} est dédiée à la création d'un entier correspondant au bouton numérique pressé. Une fois le chiffre saisi, il est directement affiché via la méthode publique \texttt{update()}. Cette approche rend le code plus modulaire et facilite l'ajout de nouvelles fonctionnalités ou d'opérations dans le futur.
\linebreak

Chaque sous-classe implémente les comportements spécifiques à une opération donnée. Cela inclut les opérateurs arithmétiques de base (\texttt{Add}, \texttt{Subtract}, \texttt{Multiply}, \texttt{Divide}) ainsi que d'autres fonctionnalités comme la gestion des pourcentages, des puissances ou des opérations spécifiques définies par l'utilisateur.
\end{flushleft}

\subsection{State}

\begin{flushleft}
	La classe \texttt{State} est responsable de la gestion des valeurs et de la pile (\texttt{Stack}). Elle joue un rôle central dans la manipulation des données et leur affichage sur l'application Calculatrice. Voici ses principales responsabilités :
\end{flushleft}

\begin{itemize}
\item \textbf{Gestion de la valeur affichée :} \\
La variable publique \texttt{value}, de type \texttt{String}, représente le contenu actuellement affiché sur l'écran de la calculatrice. Les méthodes publiques de \texttt{State} permettent d'accéder directement à cette variable et de la modifier en fonction des actions effectuées par l'utilisateur.

\item \textbf{Vérifications et gestion des erreurs :} \\
Des méthodes spécifiques permettent de vérifier si un objet existe ou si sa valeur est \texttt{null}. Cela garantit une manipulation sécurisée des données et évite les erreurs inattendues lors de l'exécution.

\item \textbf{Interaction avec la pile (\texttt{Stack}) :} \\
La pile est utilisée pour gérer les opérations complexes nécessitant un stockage temporaire de valeurs intermédiaires. Des méthodes privées sont mises en place pour permettre à la classe \texttt{State} de vérifier et de manipuler l'état de la \texttt{Stack} sans exposer directement son contenu. Cette encapsulation renforce la sécurité et la robustesse de la logique.

\item \textbf{Extensions et modularité :} \\
La classe \texttt{State} est conçue de manière extensible, permettant d'ajouter de nouvelles fonctionnalités ou de modifier le comportement existant sans compromettre la stabilité du système. Par exemple, des méthodes pourraient être ajoutées pour prendre en charge des formats d'entrée plus complexes (nombres décimaux, exposants, etc.).
\end{itemize}


	
\section{Modélisation UML}
\subsection{Diagramme des classes}
\begin{figure}[H]
	\centering
	\includegraphics[width=1.1\textwidth]{../UML_Diagrams/diagram_final.jpeg}
	\caption{Implémentation de la modélisation de la Calculatrice}
\end{figure}

\pagebreak

\section{Tests}

Les tests suivants ont été réalisés pour valider les fonctionnalités principales de la calculatrice. 

\subsection{Stack}
\begin{itemize}
	\item \textbf{Étape 1 : Empiler des valeurs et tester} \\
	Valeurs empilées : 10.0 et 5.0. \\
	Résultat attendu : \texttt{[5.0, 10.0]} \\
	Résultat obtenu : \texttt{[5.0, 10.0]} \\
	\textbf{Test réussi.}
\end{itemize}

\subsection{Addition}
\begin{itemize}
	\item \textbf{Étape 2 : Addition (10 + 5)} \\
	Opération : Addition des deux valeurs empilées. \\
	Résultat attendu : \texttt{[15.0]} \\
	Résultat obtenu : \texttt{[15.0]} \\
	\textbf{Test réussi.}
\end{itemize}

\subsection{Multiplication}
\begin{itemize}
	\item \textbf{Étape 3 : Empiler une nouvelle valeur et multiplier} \\
	Valeur ajoutée : 3.0. \\
	Opération : Multiplication (15.0 * 3.0). \\
	Résultat attendu : \texttt{[45.0]} \\
	Résultat obtenu : \texttt{[45.0]} \\
	\textbf{Test réussi.}
\end{itemize}

\subsection{Division}
\begin{itemize}
	\item \textbf{Étape 4 : Diviser par une nouvelle valeur} \\
	Valeur ajoutée : 9.0. \\
	Opération : Division (45.0 / 9.0). \\
	Résultat attendu : \texttt{[5.0]} \\
	Résultat obtenu : \texttt{[5.0]} \\
	\textbf{Test réussi.}
\end{itemize}

\subsection{Opérations avancées}
\begin{itemize}
	\item \textbf{Étape 5 : Racine carrée} \\
	Valeur ajoutée : 4.0. \\
	Opération : Racine carrée de 4.0. \\
	Résultat attendu : \texttt{[2.0, 5.0]} \\
	Résultat obtenu : \texttt{[2.0, 5.0]} \\
	\textbf{Test réussi.}
	\\
	\item \textbf{Étape 6 : Mise au carré} \\
	Opération : Mise au carré du sommet de la pile (2.0). \\
	Résultat attendu : \texttt{[4.0, 5.0]} \\
	Résultat obtenu : \texttt{[4.0, 5.0]} \\
	\textbf{Test réussi.}
	\\
	\item \textbf{Étape 7 : Inverse (1/x)} \\
	Opération : Calcul de l'inverse (1/4.0). \\
	Résultat attendu : \texttt{[0.25, 5.0]} \\
	Résultat obtenu : \texttt{[0.25, 5.0]} \\
	\textbf{Test réussi.}
	\\
	\item \textbf{Étape 8 : Combinaison complexe} \\
	Valeur ajoutée : 2.0. \\
	Opérations : Multiplication (0.25 * 2.0), ajout de 5.0, soustraction. \\
	Résultat attendu : \texttt{[-4.5, 5.0]} \\
	Résultat obtenu : \texttt{[-4.5, 5.0]} \\
	\textbf{Test réussi.}
\end{itemize}

\subsection{Réinitialisation}
\begin{itemize}
	\item \textbf{Étape 9 : Réinitialisation de la pile} \\
	Opération : Effacement complet de la pile. \\
	Résultat attendu : \texttt{[]} \\
	Résultat obtenu : \texttt{[]} \\
	\textbf{Test réussi.}
\end{itemize}

\subsection{Résumé des résultats}
Tous les tests ont été exécutés avec succès, validant ainsi les principales fonctionnalités de la calculatrice, y compris les opérations arithmétiques de base, les opérations avancées, et la gestion correcte de l'état interne.


\end{document}
