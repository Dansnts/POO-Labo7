\PassOptionsToPackage{svgnames}{xcolor}
\documentclass[12pt, a4paper, onecolumn]{article} 
\usepackage[utf8]{inputenc}
\usepackage{fancyhdr}
\usepackage{graphicx}
\usepackage{geometry}
\usepackage{float}
\usepackage{multicol} % Ajout du paquet multicol pour gérer les colonnes
\usepackage{lmodern}  % Latin Modern scalable font
\usepackage[T1]{fontenc}  % Use T1 font encoding
\usepackage{hyperref}  % Pour rendre la table des matières interactive
\usepackage{pdfpages}

% Réduire les marges
\geometry{top=1.5cm, bottom=1.5cm, left=1.5cm, right=1.5cm}

\input{structure.tex} 

%----------------------------------------------------------------------------------------
%	ARTICLE INFORMATION
%----------------------------------------------------------------------------------------

\title{Labo7 - POO \\ Calculatrice} 

% Modifiez le style des auteurs
\newcommand{\largename}[1]{{\Large\textbf{#1}}} % style pour le nom de famille

\author{
	\authorstyle{Dani Tiago \largename{Faria dos Santos}\\ Antoine \largename{Aubry } \\ \\ Groupe  \textbf{L02GrP}\\ HEIG-VD} % Authors
}

\makeatletter
\renewcommand\date[1]{\gdef\@date{\hbox to \linewidth{#1\hss}}}
\makeatother
\date{\today}


\pagestyle{fancy}
\fancyhf{}
\fancyhead[L]{Laboratoire 7 - POO}
\fancyhead[R]{Groupe L03GrD}
\fancyfoot[R]{Page \thepage}
\fancyfoot[L]{HEIG-VD | Dani Tiago \textbf{Faria dos Santos} - Antoine \textbf{Aubry }}



% Ligne sous l'en-tête
\renewcommand{\headrulewidth}{0.7pt}
% Pas de ligne sur le pied de page
\renewcommand{\footrulewidth}{0.5pt}


\begin{document}
	% Utilisation de twocolumn pour le titre
	\twocolumn[ 
	\maketitle
	]
	
	% Retour à une seule colonne
	\onecolumn 
	
	% Table des matières
	\tableofcontents
	\newpage
	
	\section{Hypothèses de travail}
	
	\section{Choix de conception}
	\subsection{Operator}
	Pour la classe Operator nous avons décider de faire une sous-classes abstraites pour chaque operation, par exemple avec Digit() qui crée un entier selon le bouton pressé et l'affiche directement avec la fonction update().
	
	\subsection{State}
	La modification des valeurs et de la Stack, s'éffectuent dans la classe state, les méthodes publiques sont celles qui permettent d'accèeder directement à la manipulation de notre variable String value, qui sera affichée sur l'applicaiton Calculatrice. Les fonction pour verifier si les objets existent ou bien si leur valeur est nulles sont implémentées dans notre classe aussi. De sméthode sprivates existement pour par exemple demande au state de vérifier l'état du Stack sans devoir passer directement par l'objet State lui même.
	
	\section{Modélisation UML}
	\subsection{Diagramme des classes}

	
	
\end{document}
