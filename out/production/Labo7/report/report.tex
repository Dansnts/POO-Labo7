\documentclass[12pt, a4paper, onecolumn]{article} 
\usepackage{pdfpages}
\usepackage[utf8]{inputenc}
\usepackage{fancyhdr}
\usepackage{graphicx}
\usepackage{geometry}
\usepackage{float}
\usepackage{multicol} % Ajout du paquet multicol pour gérer les colonnes
\usepackage{lmodern}  % Latin Modern scalable font
\usepackage[T1]{fontenc}  % Use T1 font encoding
\usepackage{hyperref}  % Pour rendre la table des matières interactive




% Réduire les marges
\geometry{top=1.5cm, bottom=1.5cm, left=1.5cm, right=1.5cm}

\input{structure.tex} 

%----------------------------------------------------------------------------------------
%	ARTICLE INFORMATION
%----------------------------------------------------------------------------------------

\title{Labo5 - POO \\ Matrices} 

% Modifiez le style des auteurs
\newcommand{\largename}[1]{{\Large\textbf{#1}}} % style pour le nom de famille

\author{
	\authorstyle{Dani Tiago \largename{Faria dos Santos}\\ Antoine \largename{Aubry } \\ \\ Groupe  \textbf{L02GrD}\\ HEIG-VD} % Authors
}

\makeatletter
\renewcommand\date[1]{\gdef\@date{\hbox to \linewidth{#1\hss}}}
\makeatother
\date{\today}


\pagestyle{fancy}
\fancyhf{}
\fancyhead[L]{Laboratoire 5 - POO}
\fancyhead[R]{Groupe L03GrD}
\fancyfoot[R]{Page \thepage}
\fancyfoot[L]{HEIG-VD | Dani Tiago \textbf{Faria dos Santos} - Antoine \textbf{Aubry }}


% Ligne sous l'en-tête
\renewcommand{\headrulewidth}{0.7pt}
% Pas de ligne sur le pied de page
\renewcommand{\footrulewidth}{0.5pt}


\begin{document}
	% Utilisation de twocolumn pour le titre
	\twocolumn[ 
	\maketitle
	]
	
	% Retour à une seule colonne
	\onecolumn 
	
	% Table des matières
	\tableofcontents
	\newpage
	
	\section{Modélisation UML}
	\begin{figure}[H]
	\centering
	\includegraphics[width=1\textwidth]{../schema.png}
	\caption{Implémentation de la modélisation de la classe Matrix}
\end{figure}
	
	\section{Choix de Conception}
	\begin{flushleft}
\begin{enumerate}
	\item \textbf{Encapsulation} : 
	La classe \texttt{Matrix} protège ses attributs (dimensions, éléments, modulo) à l'aide de modificateurs d'accès, garantissant ainsi l'intégrité des données et minimisant les accès non autorisés.
	
	\item \textbf{Gestion des Exceptions} : 
	Des exceptions (\texttt{RuntimeException}) sont lancées pour gérer les cas de paramètres non valides, comme des dimensions négatives ou des modulos différents, permettant ainsi une gestion robuste des erreurs.
	
	\item \textbf{Modélisation Orientée Objet} : 
	Les opérations arithmétiques (addition, soustraction, multiplication) sont modélisées via des classes spécifiques (par exemple, \texttt{Addition}, \texttt{Subtraction}, \texttt{Multiplication}). Cela permet une factorisation du code et une extensibilité, facilitant l'ajout de nouvelles opérations à l'avenir.
	
	\item \textbf{Gestion des Dimensions} : 
	Lors de l'opération entre matrices de tailles différentes, le résultat est une nouvelle matrice de taille maximale avec des valeurs manquantes remplacées par des zéros, ce qui assure une manipulation cohérente des données.

    \item \textbf{Centralisation du Code Commun} : 
	Le code commun aux différentes opérations est encapsulé pour éviter la duplication, ce qui permet d'ajouter facilement de nouvelles opérations sans nécessiter de structures de contrôle complexes comme \texttt{if} ou \texttt{switch}.

\end{enumerate}
\end{flushleft}
	
	\section{Listing du Code des Tests}
	\subsection{Test de la classe Matrix}

	\includepdf[pages=-]{src_test_java.pdf}
	

	\newpage
	
	\section{Documentation des Tests Effectués}
	\subsection{Tests de la Classe \texttt{Matrix}}
	
	\begin{itemize}
		\item \textbf{testAdditionOperation()}
		\begin{itemize}
			\item \textbf{Description} : Teste l'addition de deux matrices.
			\item \textbf{Entrées} : \texttt{matrix1} contenant les valeurs $\begin{bmatrix} 1 & 2 \\ 3 & 4 \end{bmatrix}$ et \texttt{matrix2} contenant $\begin{bmatrix} 5 & 6 \\ 7 & 8 \end{bmatrix}$.
			\item \textbf{Résultats Attendus} : Les éléments résultants doivent être $\begin{bmatrix} 6 & 8 \\ 10 & 12 \end{bmatrix}$.
		\end{itemize}
		
		\item \textbf{testSubtractionOperation()}
		\begin{itemize}
			\item \textbf{Description} : Teste la soustraction de deux matrices.
			\item \textbf{Entrées} : \texttt{matrix1} et \texttt{matrix2} comme ci-dessus.
			\item \textbf{Résultats Attendus} : Les éléments résultants doivent être $\begin{bmatrix} -4 & -4 \\ -4 & -4 \end{bmatrix}$.
		\end{itemize}
		
		\item \textbf{testMultiplicationOperation()}
		\begin{itemize}
			\item \textbf{Description} : Teste la multiplication de deux matrices.
			\item \textbf{Entrées} : \texttt{matrix1} et \texttt{matrix2} comme ci-dessus.
			\item \textbf{Résultats Attendus} : Les éléments résultants doivent être $\begin{bmatrix} 19 & 22 \\ 43 & 50 \end{bmatrix}$.
		\end{itemize}
		
		\item \textbf{testMatrixInitialization()}
		\begin{itemize}
			\item \textbf{Description} : Vérifie que les matrices sont correctement initialisées.
			\item \textbf{Résultats Attendus} : Les matrices ne doivent pas être nulles après initialisation.
		\end{itemize}
		
		\item \textbf{testValueSettingAndGetting()}
		\begin{itemize}
			\item \textbf{Description} : Vérifie la possibilité de définir et d'obtenir des valeurs dans la matrice.
			\item \textbf{Entrées} : Valeurs initiales et modifications.
			\item \textbf{Résultats Attendus} : Les valeurs doivent être correctement mises à jour et récupérées.
		\end{itemize}
		
		\item \textbf{testDisplay()}
		\begin{itemize}
			\item \textbf{Description} : Teste que la méthode d'affichage ne lève pas d'exceptions.
			\item \textbf{Résultats Attendus} : L'appel de la méthode d'affichage ne doit pas provoquer d'erreur.
		\end{itemize}
	\end{itemize}
		\newpage
	\subsection{Tests d'Initialisation des Matrices}
	
	\begin{itemize}
		\item \textbf{testCreateMatricesWithValidInput()}
		\begin{itemize}
			\item \textbf{Description} : Teste la création de matrices avec des entrées utilisateur valides.
			\item \textbf{Entrées} : Dimensions et valeurs définies par l'utilisateur.
			\item \textbf{Résultats Attendus} : Les matrices doivent être créées avec les dimensions et valeurs correctes.
		\end{itemize}
		
		\item \textbf{testCreateMatricesWithInvalidDimensions()}
		\begin{itemize}
			\item \textbf{Description} : Vérifie que le système gère les dimensions invalides.
			\item \textbf{Entrées} : Dimensions négatives.
			\item \textbf{Résultats Attendus} : Une exception doit être levée avec un message approprié.
		\end{itemize}
		
		\item \textbf{testCreateMatricesWithDifferentModulos()}
		\begin{itemize}
			\item \textbf{Description} : Teste la création de matrices avec des modulos différents.
			\item \textbf{Entrées} : Modulos non identiques.
			\item \textbf{Résultats Attendus} : Une exception doit être levée avec un message indiquant que les modulos sont différents.
		\end{itemize}
	\end{itemize}
	
	
	
	
\end{document}
